\chapter{Load Balancer and Virtual Host \\
\small{\textit{-- Thomas Ung}}
\index{load balancer}
\index{virtual host}
\index{Chapter!Load Balancer and Virtual Host}
\label{Chapter::LoadBalancer}}

\section{Overview}

This section documents the creation and deployment of a simple load-balanced web application using Docker and Nginx on a DigitalOcean Ubuntu 22.04 Droplet.  
The objective is to demonstrate two concepts: (1) a round-robin load balancer that alternates traffic between two web servers, and (2) name-based virtual hosting that routes different subdomains to unique back-end containers.

\section{Environment Setup}

\begin{enumerate}
    \item \textbf{Create the Droplet:}
    \begin{itemize}
        \item OS: Ubuntu 22.04 LTS  
        \item Size: 2 vCPUs, 4 GB RAM  
        \item Public IP Address: \texttt{167.99.54.162}
    \end{itemize}

    \item \textbf{Install Docker and Docker Compose:}

    \begin{minted}[fontsize=\small,breaklines]{bash}
sudo apt update && sudo apt upgrade -y
sudo apt install docker.io docker-compose -y
sudo systemctl enable docker
sudo systemctl start docker
docker --version
docker compose version
    \end{minted}

    \item \textbf{Verify Docker:}

    \begin{minted}[fontsize=\small,breaklines]{bash}
docker info
    \end{minted}
\end{enumerate}

\section{Directory Structure}

\begin{minted}[fontsize=\small,breaklines]{text}
load-balanced-app/
│
├── docker-compose.yml
├── nginx/
│   └── nginx.conf
├── web1/
│   └── index.html
└── web2/
    └── index.html
\end{minted}

\section{Web Server Files}

Each web server container serves static HTML content:

\begin{minted}[fontsize=\small,breaklines]{html}
<!-- web1/index.html -->
<h1>Hello from Web 1</h1>
\end{minted}

\begin{minted}[fontsize=\small,breaklines]{html}
<!-- web2/index.html -->
<h1>Hello from Web 2</h1>
\end{minted}

\section{Nginx Configuration}

\subsection{Combined Load Balancer and Virtual Host Configuration}

\begin{minted}[fontsize=\small,breaklines]{nginx}
events {}

http {
    upstream backend {
        server web1:80;
        server web2:80;
    }

    # Load Balancer
    server {
        listen 80;
        server_name loadbalancer.quackops.me;

        location / {
            proxy_pass http://backend;
            proxy_set_header Host $host;
            proxy_set_header X-Real-IP $remote_addr;
        }
    }

    # Virtual Host 1
    server {
        listen 80;
        server_name web1.quackops.me;

        location / {
            proxy_pass http://web1:80;
            proxy_set_header Host $host;
            proxy_set_header X-Real-IP $remote_addr;
        }
    }

    # Virtual Host 2
    server {
        listen 80;
        server_name web2.quackops.me;

        location / {
            proxy_pass http://web2:80;
            proxy_set_header Host $host;
            proxy_set_header X-Real-IP $remote_addr;
        }
    }
}
\end{minted}

\section{Docker Compose File}

\begin{minted}[fontsize=\small,breaklines]{yaml}
version: '3'
services:
  web1:
    image: nginx
    container_name: web1
    volumes:
      - ./web1:/usr/share/nginx/html:ro

  web2:
    image: nginx
    container_name: web2
    volumes:
      - ./web2:/usr/share/nginx/html:ro

  loadbalancer:
    image: nginx
    container_name: loadbalancer
    ports:
      - "80:80"
    volumes:
      - ./nginx/nginx.conf:/etc/nginx/nginx.conf:ro
    depends_on:
      - web1
      - web2
\end{minted}

\section{Deployment Steps}

\begin{enumerate}
    \item \textbf{Copy Project Files to Droplet:}

    \begin{minted}[fontsize=\small,breaklines]{bash}
scp -r load-balanced-app/* root@167.99.54.162:/home/load-balanced-app/
    \end{minted}

    \item \textbf{Start Containers:}

    \begin{minted}[fontsize=\small,breaklines]{bash}
cd /home/load-balanced-app
docker compose up -d --build
docker ps
    \end{minted}

    Verify that the containers \texttt{web1}, \texttt{web2}, and \texttt{loadbalancer} are running.
\end{enumerate}

\section{DNS Configuration}

The following A records were created in Namecheap:

\begin{center}
\begin{tabular}{|l|l|l|}
\hline
\textbf{Host} & \textbf{Type} & \textbf{Value (IP)} \\
\hline
@ & A & 167.99.54.162 \\
loadbalancer & A & 167.99.54.162 \\
web1 & A & 167.99.54.162 \\
web2 & A & 167.99.54.162 \\
\hline
\end{tabular}
\end{center}

\section{Testing and Verification}

\subsection{Load Balancer}

Visit:
\begin{center}
\textbf{\url{http://loadbalancer.quackops.me}}
\end{center}

Refreshing alternates between:
\begin{itemize}
    \item \texttt{Hello from Web 1}
    \item \texttt{Hello from Web 2}
\end{itemize}

\subsection{Virtual Hosting}

\begin{itemize}
    \item \textbf{Web 1:} \url{http://web1.quackops.me} → Displays ``Hello from Web 1''  
    \item \textbf{Web 2:} \url{http://web2.quackops.me} → Displays ``Hello from Web 2''  
\end{itemize}

\section{Troubleshooting Notes}

\begin{itemize}
    \item \textbf{403 Forbidden:} Ensure the \texttt{events \{\}} block exists at the top of \texttt{nginx.conf}.  
    \item \textbf{NXDOMAIN Errors:} Wait for DNS propagation or verify that each subdomain points to \texttt{167.99.54.162}.  
    \item \textbf{Port Conflicts:} Confirm that no other service is using port 80.  
    \item \textbf{Connection Persistence:} If load balancing does not alternate, add the directive \texttt{random;} inside the \texttt{upstream} block.  
\end{itemize}

\section{Final Result}

The Nginx reverse proxy successfully provides both load balancing and name-based virtual hosting.  
All components operate as expected, with subdomains resolving correctly through Namecheap DNS.

\begin{center}
\textbf{\url{http://loadbalancer.quackops.me}} \\
\textbf{\url{http://web1.quackops.me}} \\
\textbf{\url{http://web2.quackops.me}}
\end{center}

\begin{figure}[H]
    \centering
    \includegraphics[width=0.7\linewidth]{png/loadbalancer.png}
    \caption{Load Balancer and Virtual Host deployment on DigitalOcean droplet (\texttt{167.99.54.162}).}
    \label{fig:loadbalancer}
\end{figure}

To verify that the load balancer was correctly alternating between the two web servers without relying on browser caching or DNS propagation, the \texttt{curl} command was used in PowerShell.  
Unlike a browser, \texttt{curl} allows direct inspection of HTTP responses and headers, making it an ideal tool for confirming the underlying server response pattern.

Each repeated request to \url{http://loadbalancer.quackops.me} returned content from alternating web containers, confirming that round-robin load balancing was functioning correctly.

\begin{minted}[fontsize=\small,breaklines]{powershell}
PS C:\Users\thoma> curl http://loadbalancer.quackops.me

StatusCode        : 200
StatusDescription : OK
Content           : <h1>Hello from Web 1 </h1>
RawContent        : HTTP/1.1 200 OK
                    Content-Type: text/html
                    Content-Length: 26
                    Date: Wed, 12 Nov 2025 03:38:53 GMT

PS C:\Users\thoma> curl http://loadbalancer.quackops.me

StatusCode        : 200
StatusDescription : OK
Content           : <h1>Hello from Web 2 </h1>
RawContent        : HTTP/1.1 200 OK
                    Content-Type: text/html
                    Content-Length: 26
                    Date: Wed, 12 Nov 2025 03:38:54 GMT

PS C:\Users\thoma> curl http://loadbalancer.quackops.me

StatusCode        : 200
StatusDescription : OK
Content           : <h1>Hello from Web 1 </h1>
RawContent        : HTTP/1.1 200 OK
                    Content-Type: text/html
                    Content-Length: 26
                    Date: Wed, 12 Nov 2025 03:38:55 GMT

PS C:\Users\thoma> curl http://loadbalancer.quackops.me

StatusCode        : 200
StatusDescription : OK
Content           : <h1>Hello from Web 2 </h1>
RawContent        : HTTP/1.1 200 OK
                    Content-Type: text/html
                    Content-Length: 26
                    Date: Wed, 12 Nov 2025 03:39:01 GMT
\end{minted}

The alternating responses of “\texttt{Hello from Web 1}” and “\texttt{Hello from Web 2}” confirm that Nginx is successfully distributing incoming requests between the two backend containers in a round-robin manner.  
This demonstrates that the configuration defined in the \texttt{upstream backend} block of \texttt{nginx.conf} is working exactly as intended.

