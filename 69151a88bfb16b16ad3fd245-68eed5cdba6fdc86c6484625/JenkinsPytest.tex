\chapter{Jenkins with Pytest \\
\small{\textit{-- Justin Baumann, Gianna Cerbone, Thomas Ung, Spurthi Setty}} 
\index{Chapter!jenkinssetup}
\index{jenkinssetup}
\label{Chapter::jenkinssetup}}

% Add a section and label it so that we can reference it later
\section{Jenkins Setup \label{Section:jenkinssetup}}
This is the order of operations executed to set up Jenkins.
\begin{enumerate}
  \item \textbf{Project directory (host):}
  \begin{verbatim}
  C:\Users\minij\Desktop\jenkins-python-pytest-demo
  \end{verbatim}

  \item \textbf{Compose file (host)} to run Jenkins in Docker:
  \begin{verbatim}
  version: '3.8'
  services:
    jenkins:
      image: jenkins/jenkins:lts
      container_name: jenkins
      ports:
        - "8080:8080"
        - "50000:50000"
      volumes:
        - jenkins_home:/var/jenkins_home
        - /var/run/docker.sock:/var/run/docker.sock
  volumes:
    jenkins_home:
  \end{verbatim}

  \item \textbf{Start Jenkins (host terminal):}
  \begin{verbatim}
  docker compose up -d
  \end{verbatim}

  \item \textbf{Access and unlock Jenkins:}
  \begin{itemize}
    \item Open \texttt{http://localhost:8080}.
    \item Retrieve initial password (host):
\begin{verbatim}
docker exec -it jenkins cat /var/jenkins_home/secrets/initialAdminPassword
\end{verbatim}
    \item Install ``Suggested plugins'' and create the admin user.
  \end{itemize}

  \item \textbf{Install Python once inside the Jenkins container} (so the agent can create a venv):
\begin{verbatim}
docker exec -u root -it jenkins bash
apt-get update
apt-get install -y python3 python3-venv python3-pip
exit
\end{verbatim}
\end{enumerate}

\subsection*{B. Source Control (GitHub)}
\begin{enumerate}
  \item Initialize and push the demo project to GitHub:
\begin{verbatim}
git init
git add .
git commit -m "Initial commit"
git branch -M main
git remote add origin https://github.com/JustBaumann/jenkins-python-pytest-demo.git
git push -u origin main
\end{verbatim}
\end{enumerate}

\subsection*{C. Jenkins Pipeline Job}
\begin{enumerate}
  \item In Jenkins: \textbf{New Item} $\rightarrow$ \textbf{Pipeline}.
  \item Under \textbf{Pipeline}:
    \begin{itemize}
      \item \textbf{Definition}: Pipeline script from SCM.
      \item \textbf{SCM}: Git.
      \item \textbf{Repository URL}: \texttt{https://github.com/JustBaumann/jenkins-python-pytest-demo.git}
      \item \textbf{Branch}: \texttt{main}
      \item \textbf{Script Path}: \texttt{Jenkinsfile}
    \end{itemize}
  \item Save, then \textbf{Build Now}.
\end{enumerate}

\subsection*{D. Pytest: What the Pipeline Does}
\noindent The pipeline runs in four stages:
\begin{enumerate}
  \item \textbf{Prepare/Install dependencies}: Create a Python virtual environment and install \texttt{pytest}.
  \item \textbf{Run tests}: Execute the unit tests and write a JUnit XML report (\texttt{report.xml}).
  \item \textbf{Publish Report}: Archive and display test results in Jenkins.
  \item (Optional) \textbf{Always publish}: ensure results are shown even when tests fail.
\end{enumerate}

\subsection*{E. Jenkinsfile Used}
\noindent Stored at the repo root as \texttt{Jenkinsfile}. It assumes Python is installed in the Jenkins container (Step~A.5).
\begin{verbatim}
pipeline {
  agent any

  stages {
    stage('Install dependencies') {
      steps {
        sh 'python3 -m venv venv'
        sh './venv/bin/pip install -r requirements.txt'
      }
    }

    stage('Run tests') {
      steps {
        // Write JUnit XML so Jenkins can publish a Test Result tab
        sh './venv/bin/pytest --junitxml=report.xml'
        // Optional to keep pipeline going even on failures:
        // sh './venv/bin/pytest --junitxml=report.xml || true'
      }
    }

    stage('Publish Report') {
      steps {
        junit 'report.xml'
      }
    }
  }
}
\end{verbatim}

\subsection*{F. Test Suite Contents}
\noindent Minimal demo tests to show a mix of pass/fail:
\begin{verbatim}
# repo path: tests/test_sample.py
def test_addition():
    assert 1 + 1 == 2

def test_subtraction():
    assert 5 - 2 == 3

def test_failure_example():
    assert 2 * 2 == 5  # intentional fail
\end{verbatim}

\subsection*{G. Results and Evidence}
\begin{itemize}
  \item After a build completes, Jenkins shows a \textbf{Test Result} link for the run
        (summary: 3 tests, 1 failed), and a \textbf{Stage/Pipeline} view with the three stages.
  \begin{figure}[h!]
    \centering
    \includegraphics[width=1.0\textwidth]{png/Jenkins Job.png}
    \caption{Screenshot of Stage/Pipeline view}
    \label{fig:Jenkins Job}
\end{figure}
\begin{figure}[h!]
    \centering
    \includegraphics[width=1.0\textwidth]{png/JenkinsTest.png}
    \caption{Screenshot of Test Result page}
    \label{fig:Jenkins Test}
\end{figure}
\end{itemize}